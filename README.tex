\documentclass[article]{jss}
\usepackage[utf8]{inputenc}

\author{
Matthieu Stigler\\UC Davis \And Bastiaan Quast\\The Graduate Institute, Geneva
}
\title{\pkg{rddtools}: tools for Regression Discontinuity Design in
R\thanks{This research was financed by the Swiss National Science Foundation (SNSF) under the grant 'Development Aid and Social Dynamics' (100018-140745) administered by the Graduate Institute's Centre on Conflict, Development and Peacebuilding (CCDP) and led by Jean-Louis Arcand. We thank Sandra Reimann and Oliver Jutersonke at the CCDP for their generous support.}}
\Keywords{RDD, Regression, Discontinuity, Design, \proglang{R}}

\Abstract{
The rddtools package provides a framework for Regression Discontinuity
Design (RDD) in R. In addtion to bringing together functionality from
several different existing package, new functionality is implemented in
terms of design and sensitivity test, as well as non parametric RDD.
}

\Plainauthor{Matthieu Stigler, Bastiaan Quast}
\Plaintitle{rddtools: tools for Regression Discontinuity Design in R}
\Shorttitle{\pkg{rddtools}}
\Plainkeywords{RDD, Regression, Discontinuity, Design R}

%% publication information
%% \Volume{50}
%% \Issue{9}
%% \Month{June}
%% \Year{2012}
\Submitdate{2015-07-25}
%% \Acceptdate{2012-06-04}

\Address{
    Matthieu Stigler\\
  UC Davis\\
  California\\
  E-mail: \href{mailto:matthieu.stigler@gmail.com}{\nolinkurl{matthieu.stigler@gmail.com}}\\
  URL: \url{https://matthieustigler.github.io/}\\~\\
      Bastiaan Quast\\
  The Graduate Institute, Geneva\\
  Maison de la paix Geneva, Switzerland\\
  E-mail: \href{mailto:bquast@gmail.com}{\nolinkurl{bquast@gmail.com}}\\
  URL: \url{http://qua.st/}\\~\\
  }

\usepackage{amsmath}

\begin{document}

\section{Introduction}\label{introduction}

The \texttt{rddtools} package attempts to provide a unified approach to
the application of Regression Discontinuity Design (RDD) in R.
Functionality from several existing packages is brought together under
one coherent API. Additionallity, the \texttt{rddtools} package
implements new functionality in several aspects of regression
discontinuity design.

\section{Design}\label{design}

A unified framework for RDD is implemented through the
\texttt{rdd\_data} class. This class inherits from the \texttt{R}
\texttt{base} package's \texttt{data.frame} class. This functionality is
made accessible through the associated \texttt{rdd\_data()} function, as
well as the following associated methods.

\begin{itemize}
\itemsep1pt\parskip0pt\parsep0pt
\item
  \texttt{{[}.rdd\_data()} / \texttt{subset.rdd\_data()}
\item
  \texttt{summary.rdd\_data()}
\item
  \texttt{plot.rdd\_data()}
\end{itemize}

The package is designed to leveredge of existing implementations of
\textbf{Regression Discontinuity Design} in \texttt{R}, such as the
\texttt{rdd} (Dimmery 2013) and \texttt{KernSmooth} (M. Wand 2015)
packages. Furthermore, general algorithms such as non-parametric
regression from the \texttt{np} package (Hayfield and Racine 2008) is
made accessible for RDD through the \texttt{rdd\_data} framework.

In addition to this, it implements several tools for RDD analysis that
were previously unavailable.

\subsection{Bandwidth Selection}\label{bandwidth-selection}

The \texttt{rddtools} package implements two new methods for bandwidth
selection. The first is the MSE-RDD bandwidth procedure of (G. Imbens
and Kalyanaraman 2012). This procdure is implemented in the
\texttt{rdd\_bw\_ik()} function. Secondly, the MSE global bandwidth
procedure of (Ruppert, Sheather, and Wand 1995) is implemented in the
\texttt{rdd\_bw\_rsw()} function.

\subsection{Estimation}\label{estimation}

Various types of RDD estimation are supported. The functionality has
been implemented in such a way, that the change from one estimation
method to another is as small as possible.

Firstly, RDD parametric estimation though the \texttt{rdd\_reg\_lm()}
function is implemented. The \texttt{rdd-reg\_lm()} fnction includes
functionality for specifying the polynomial order, including covariates
with various specifications as advocated in (G. W. Imbens and Lemieux
2008).

Secondly, RDD local non-parametric estimation is available by means of
the \texttt{rdd\_reg\_np()} function. The \texttt{rdd\_reg\_np()}
function can also include covariates and allows different types of
inference ( such as fully non-parametric, or parametric approximation).

Lastly, RDD generalised estimation has been implemented. This allows to
use custom estimating functions to get the RDD coefficient. For example
a probit RDD, or quantile regression could be used here.

\subsection{Post-Estimation}\label{post-estimation}

A collection of Post-Estimation tools allow the robustness of the
estimation results to be verified.

This includes various tools, such as the \texttt{rdd\_pred()}, which is
used to obtain predictions at given covariate values. As well as the
\texttt{as.lm()} function, which is used to convert to the lm class.
Furthermore there is the \texttt{as.npreg()} function, in order to
convert to the \texttt{np} package.

Additional post-estmimation tools include \texttt{clusterInf()}, which
can be used for inference with clustered data, using eithera covariance
matrix (using the \texttt{vcovCluster()} function), or by a degrees of
freedom corrention (as described in (Cameron, Gelbach, and Miller
2008)).

Finally, the package contains functions to replicate the Monte-Carlo
simulations of {[}Imbens and Kalyanaraman 2012{]}, using the
\texttt{gen\_mc\_ik()} function.

\subsection{Regression Sensitivity
Analysis}\label{regression-sensitivity-analysis}

Regression sensitivity analysis can be conducted using the
\texttt{plotSense()} function, which test the sensitivity of the
coefficient with respect to the bandwidth, or by means of
\textbf{Placebo plot} using different cutpoints: \texttt{plotPlacebo()}

\subsection{Design sensitivity
analysis}\label{design-sensitivity-analysis}

Finally, methods for design sensitivity analysis are included.

The McCrary test of manipulation of the forcing variable is available by
passing the wrapper \texttt{dens\_test()} to the function
\texttt{DCdensity()} from package \texttt{rdd}.

As well as, the test of equal means of covariates, which can be
performed using the \texttt{covarTest\_mean()} function.

In addition to this, the test of equal density of covariates is
available via the \texttt{covarTest\_dens()} function.

\section{Data}\label{data}

A collection of typical data sets is included in the package.

\begin{itemize}
\itemsep1pt\parskip0pt\parsep0pt
\item
  Initiative Nationale pour le Développement Humain (Arcand, Rieger, and
  Nguyen 2015): \texttt{indh}
\item
  Voting in the U.S. House of Representatives (Lee 2008): \texttt{house}
\item
  STAR dataset (Angrist and Pischke 2008): \texttt{STAR\_MHE}
\end{itemize}

All three data sets are made available as \texttt{data.frame} objects.
Using the previously discussed \texttt{rdd\_data()} function we can
transform such a \texttt{data.frame} object to an object of class
\texttt{rdd\_data} (which inherits from the \texttt{data.frame} object
class).

Here we take the data set from the Initiative Nationale pour le
Développement Humain (INDH), a development project in Morocco. The data
is included with the package under the name \texttt{indh}.

\begin{CodeChunk}
\begin{CodeOutput}
Warning: package 'car' was built under R version 3.2.2
\end{CodeOutput}
\begin{CodeOutput}
[1] "indh"
\end{CodeOutput}
\end{CodeChunk}

After having loaded the data, we start with inspecting it's structure.

\begin{CodeChunk}
\begin{CodeInput}
str(indh)
\end{CodeInput}
\begin{CodeOutput}
'data.frame':   720 obs. of  2 variables:
 $ choice_pg: int  0 1 1 1 1 1 0 1 0 0 ...
 $ poverty  : num  30.1 30.1 30.1 30.1 30.1 ...
\end{CodeOutput}
\end{CodeChunk}

The \texttt{indh} object is a \texttt{data.frame} containing 720
observations (representing individuals) of two variables:

\begin{itemize}
\itemsep1pt\parskip0pt\parsep0pt
\item
  \texttt{choice\_pg}
\item
  \texttt{poverty}
\end{itemize}

The variable of interest is \texttt{choice\_pg}, which represent the
decision to contribute to a public good or not. The observations are
individuals choosing to contribute or not, these individuals are
clustered by the variable \texttt{commune} which is the municipal
structure at which funding was distributed as part of the INDH project.
The forcing variable is \texttt{poverty} which represents the number of
households in a commune living below the poverty threshold. As part of
the INDH, commune with a proportion of household below the poverty
threshold greater than 30\% were allowed to distribute the funding using
a \textbf{Community Driven Development} scheme. The cutoff point for our
analysis is therefore \texttt{30}.

We can now transform the \texttt{data.frame} to a special
\texttt{rdd\_data}-class object, inheriting from the \texttt{data.frame}
class using the \texttt{rdd\_data()} function.

\begin{CodeChunk}
\begin{CodeInput}
rdd_dat_indh <- rdd_data(y=choice_pg,
                         x=poverty,
                         data=indh,
                         cutpoint=30 )
\end{CodeInput}
\end{CodeChunk}

The \texttt{rdd\_data()} can be used using the \texttt{data} argument,
in which case the function will look for the values of \texttt{y} and
\texttt{x} in this argument (before looking in the \texttt{.GlobalEnv}),
if this argument is \texttt{NULL}, only the \texttt{.GlobalEnv} will be
scanned. Additional exogenous variables can be included using the
\texttt{covar} argument.

The structure is similar to the original \texttt{data.frame} object, but
contains some additional information.

\begin{CodeChunk}
\begin{CodeInput}
str(rdd_dat_indh)
\end{CodeInput}
\begin{CodeOutput}
Classes 'rdd_data' and 'data.frame':    720 obs. of  2 variables:
 $ x: num  30.1 30.1 30.1 30.1 30.1 ...
 $ y: int  0 1 1 1 1 1 0 1 0 0 ...
 - attr(*, "hasCovar")= logi FALSE
 - attr(*, "labels")= list()
 - attr(*, "cutpoint")= num 30
 - attr(*, "type")= chr "Sharp"
\end{CodeOutput}
\end{CodeChunk}

The \texttt{rdd\_data} object has the classes \texttt{data.frame} and
\texttt{rdd\_data}. It contains two variables, \texttt{y} the
explanandum or dependent variable and \texttt{x} the explanans or
driving variable, which is also our discontinuous variable. Related to
the discontinuous variable is the \texttt{attribute} called
\texttt{cutpoint}, which describes where in the domain of \texttt{x} the
discontinuity occurs. The \texttt{hasCover} attribute indicates if
additional exogenous variables have been included using the
\texttt{cover} argument to the \texttt{rdd\_data()} function.

\section{Analysis}\label{analysis}

In order to best understand our data, we start with an exploratory data
analysis using tables\ldots{}

\begin{CodeChunk}
\begin{CodeInput}
summary(rdd_dat_indh)
\end{CodeInput}
\begin{CodeOutput}
### rdd_data object ###

Cutpoint: 30 
Sample size: 
    -Full : 720 
    -Left : 362 
    -Right: 358
Covariates: no 
\end{CodeOutput}
\end{CodeChunk}

\ldots{}and plots.

\begin{CodeChunk}
\begin{CodeInput}
plot(rdd_dat_indh)
\end{CodeInput}


\begin{center}\includegraphics{README_files/figure-latex/plot-1} \end{center}

\end{CodeChunk}

\subsection{Parametric Estimation}\label{parametric-estimation}

We can now continue with a standard Regression Discontinuity Design
estimation.

\begin{CodeChunk}
\begin{CodeInput}
reg_para <- rdd_reg_lm(rdd_dat_indh, order=4)
print(reg_para) # uses print.rdd_data
\end{CodeInput}
\begin{CodeOutput}
### RDD regression: parametric ###
    Polynomial order:  4 
    Slopes:  separate 
    Number of obs: 720 (left: 362, right: 358)

    Coefficient:
  Estimate Std. Error t value Pr(>|t|)
D  0.22547    0.17696  1.2741    0.203
\end{CodeOutput}
\end{CodeChunk}

and visualising this estimation.

\begin{CodeChunk}
\begin{CodeInput}
plot(reg_para)
\end{CodeInput}


\begin{center}\includegraphics{README_files/figure-latex/plot-reg_para-1} \end{center}

\end{CodeChunk}

\subsection{Non-parametric Estimation}\label{non-parametric-estimation}

In addition to the parametric estimation, we can also perform a
non-parametric estimation.

\begin{CodeChunk}
\begin{CodeInput}
bw_ik <- rdd_bw_ik(rdd_dat_indh)
reg_nonpara <- rdd_reg_np(rdd_object=rdd_dat_indh, bw=bw_ik)
reg_nonpara
\end{CodeInput}
\begin{CodeOutput}
### RDD regression: nonparametric local linear###
    Bandwidth:  0.790526 
    Number of obs: 460 (left: 139, right: 321)

    Coefficient:
  Estimate Std. Error z value Pr(>|z|)
D 0.144775   0.095606  1.5143     0.13
\end{CodeOutput}
\end{CodeChunk}

and visualising the non-parametric estimation.

\begin{CodeChunk}
\begin{CodeInput}
plot(reg_nonpara)
\end{CodeInput}


\begin{center}\includegraphics{README_files/figure-latex/plot-reg_nonpara-1} \end{center}

\end{CodeChunk}

\subsection{Sensitivity tests.}\label{sensitivity-tests.}

In addition to this, sepeveral sensitivity tests for the parametric and
non-parametric estimation methods have been implemented.

\begin{CodeChunk}
\begin{CodeInput}
plotSensi(reg_nonpara, from=0.05, to=1, by=0.1)
\end{CodeInput}


\begin{center}\includegraphics{README_files/figure-latex/sensi-1} \end{center}

\end{CodeChunk}

In addition to the sensitivity test, we can also perform various other
test such as a placebo test.

\section{Conclusion and Discussion}\label{conclusion-and-discussion}

The package \texttt{rddtools} provides a unified framework for working
with Regression Discontinuity Data in \texttt{R}. Functionality already
available is several existing packages, such as \texttt{rdd} and
\texttt{KernSmooth} can now easily be utilised using the
\texttt{rdd\_data} framework, as well as several linking functions.

In addition to this, new tools and algorithms have also been implemented
Furthermore, various post-estimation robustness checks are also included
in the package.

In addition to the various procedures discussed here, future packages
implementing further RDD functionality can easily leverage the
\texttt{rdd\_data} framework, which will allow users to quickly access
this new functionality through a familiar API.

\section*{References}\label{references}
\addcontentsline{toc}{section}{References}

Angrist, Joshua D, and J{ö}rn-Steffen Pischke. 2008. \emph{Mostly
Harmless Econometrics: An Empiricist's Companion}. Princeton university
press.

Arcand, Rieger, and Nguyen. 2015. ``Development Aid and Social Dyanmics
Data Set.''

Cameron, A Colin, Jonah B Gelbach, and Douglas L Miller. 2008.
``Bootstrap-Based Improvements for Inference with Clustered Errors.''
\emph{The Review of Economics and Statistics} 90 (3). MIT Press:
414--27.

Dimmery, Drew. 2013. \emph{Rdd: Regression Discontinuity Estimation}.
\url{http://CRAN.R-project.org/package=rdd}.

Hayfield, Tristen, and Jeffrey S. Racine. 2008. ``Nonparametric
Econometrics: The Np Package.'' \emph{Journal of Statistical Software}
27 (5). \url{http://www.jstatsoft.org/v27/i05/}.

Imbens, Guido W, and Thomas Lemieux. 2008. ``Regression Discontinuity
Designs: A Guide to Practice.'' \emph{Journal of Econometrics} 142 (2).
Elsevier: 615--35.

Imbens, Guido, and Karthik Kalyanaraman. 2012. ``Optimal Bandwidth
Choice for the Regression.''

Lee, David S. 2008. ``Randomized Experiments from Non-Random Selection
in US House Elections.'' \emph{Journal of Econometrics} 142 (2).
Elsevier: 675--97.

Ruppert, David, Simon J Sheather, and Matthew P Wand. 1995. ``An
Effective Bandwidth Selector for Local Least Squares Regression.''
\emph{Journal of the American Statistical Association} 90 (432). Taylor
\& Francis: 1257--70.

Wand, Matt. 2015. \emph{KernSmooth: Functions for Kernel Smoothing
Supporting Wand \& Jones (1995)}.
\url{http://CRAN.R-project.org/package=KernSmooth}.

Wickham, Hadley. 2009. \emph{Ggplot2: Elegant Graphics for Data
Analysis}. Springer New York. \url{http://had.co.nz/ggplot2/book}.

Zeileis, Achim. 2004. ``Econometric Computing with HC and HAC Covariance
Matrix Estimators. Journal of Statistical Software.'' \emph{Journal of
Statistical Software} 11 (10). \url{http://www.jstatsoft.org/v11/i10/}.

---------. 2006. ``Object-Oriented Computation of Sandwich Estimators''
16 (9). Nournal os Statistical Software.
\url{http://www.jstatsoft.org/v16/i09/}.

\end{document}

